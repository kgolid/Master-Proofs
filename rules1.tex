\clearpage
\section{Rules}
\label{sec:Rules}
\subsection{Specific Rules}
\label{sub:Specific Rules}
\subsubsection{Path Shortening Rules (S)}
\label{subs:Path Shortening Rule}
\[
\begin{prooftree}
  \Hypo{P^o_o(a,c),E(a,B)}
  \Infer1[S1]{ \bigvee_{b \in B}P^e_e(b,c)}
\end{prooftree}
\quad \quad \quad
\begin{prooftree}
  \Hypo{P^e_o(a,c)}
  \Infer1[S2]{ P^o_e(b,c),E(a,\{b\})}
\end{prooftree}
\]

\[
\begin{prooftree}[downwards]
  \Hypo{a=c}
  \Hypo{\bigvee_{b \in B}P^e_o(b,c)}
  \Infer2[S3]{P^o_e(a,c),E(a,B)}
\end{prooftree}
\quad \quad \quad
\begin{prooftree}[downwards]
  \Hypo{a=c}
  \Hypo{P^o_o(b,c),E(a,\{b\})}
  \Infer2[S4]{ P^e_e(a,c)}
\end{prooftree}
\]

\subsubsection{Path Composition Rules (C)}
\label{subs:Path Composition Rules (C)}
\[
\begin{prooftree}
  \Hypo{P^o_o(a,b),P^e(b,c)}
  \Infer1[C1]{P^o(a,c)}
\end{prooftree}
\quad \quad \quad
\begin{prooftree}
  \Hypo{P^e_o(a,b),P^o(b,c)}
  \Infer1[C2]{P^e(a,c)}
\end{prooftree}
\]

\[
\begin{prooftree}
  \Hypo{P^o_e(a,b),P^o(b,c)}
  \Infer1[C3]{P^o(a,c)}
\end{prooftree}
\quad \quad \quad
\begin{prooftree}
  \Hypo{P^e_e(a,b),P^e(b,c)}
  \Infer1[C4]{P^e(a,c)}
\end{prooftree}
\]

\subsubsection{Brading Rules (B)}
\label{subs:Brading Rules}
\[
\begin{prooftree}[downwards]
  \Hypo{P^e_e(b,c)}
  \Hypo{P^o_e(c,b)}
  \Infer2[B1]{P^e_o(a,b), P^o_o(a,c)}
\end{prooftree}
\quad \quad \quad
\begin{prooftree}[downwards]
  \Hypo{P^e_o(b,c)}
  \Hypo{P^e_o(c,b)}
  \Infer2[B2]{P^e_o(a,b), P^o_e(a,c)}
\end{prooftree}
\]

\[
\begin{prooftree}[downwards]
  \Hypo{P^o_o(b,c)}
  \Hypo{P^o_o(c,b)}
  \Infer2[B3]{P^e_e(a,b), P^o_o(a,c)}
\end{prooftree}
\quad \quad \quad
\begin{prooftree}[downwards]
  \Hypo{P^o_e(b,c)}
  \Hypo{P^e_e(c,b)}
  \Infer2[B4]{P^e_e(a,b), P^o_e(a,c)}
\end{prooftree}
\]
TODO: All of these will need justification.
\subsection{General Rules}
\label{sub:General Rules}
We can use variables to bring down the number of rules.
There are several ways of doing this, but this way will show to be the appropriate for our intentions.
\subsubsection{General Path Composition (GC)}
\label{subs:Path Composition}
\[
\begin{prooftree}
  \Hypo{\pox(a,b),P^{\overline{x}}(b,c)}
  \Infer1[GC1]{P^o(a,c)}
\end{prooftree}
\quad \quad \quad
\begin{prooftree}
  \Hypo{\pex(a,b),P^x(b,c)}
  \Infer1[GC2]{P^e(a,c)}
\end{prooftree}
\]

Notice that GC1 covers C1 and C3 while GC2 covers C2 and C4.
\subsubsection{General Brading (GB)}
\label{subs:General Brading (GB)}
\[
\begin{prooftree}[downwards]
  \Hypo{\p ox(b,c)}
  \Hypo{\p xx(c,b)}
  \Infer2[GB1]{\p ee(a,b),\p ox(a,c)}
\end{prooftree}
\quad \quad \quad
\begin{prooftree}[downwards]
  \Hypo{\penx(b,c)}
  \Hypo{\p x{\overline{x}}(c,b)}
  \Infer2[GB2]{\peo(a,b), \pox(a,c)}
\end{prooftree}
\]

GB1 covers B3 and B4 while GB2 covers B1 and B2.
