\section{Binary NANDs}
\label{sec:Binary NANDs}
\begin{definition}
  A NAND-clause is \textbf{binary} if it consists of only one or two atoms\todo{atoms?}.
\end{definition}
\begin{definition}
  A NAND-clause is \textbf{binary-derivable} if it is provable in our refutation system using only other binary clauses.
\end{definition}
\begin{definition}
  A NAND-clause is \textbf{purely binary} if it is both binary and binary-derivable.
\end{definition}
The main goal in this section will be to answer the following question:
Given a purely binary NAND-clause $\ol{ab}$, how are the nodes $a$ and $b$ structurally related in the corresponding graph?
In order to answer this question, let us first get some intuition behind this peculiar subgroup of clauses.  Here are some immediate observations:
\begin{lemma}
  All axiomatic NAND-clauses are purely binary.
\end{lemma}
\begin{proof}
  The axiomatic NAND-clauses require no proofs, so they are vacuously binary-derivable.
  Since they are all binary, they are by definition purely binary.
\end{proof}
Since these axioms corresponds to \textit{edges} in the graph, we can already conclude that whatever graph relation we are looking for, it needs to hold for adjacent nodes (i.e. nodes with an edge between them).
\begin{lemma}
  Given any instance of the $(Rneg)$-rule, the clause in the conclusion is binary-derivable if and only if all the clauses in the premise are purely binary.
\end{lemma}
\begin{proof}
  $(\Rightarrow)$ Given an instance of the $(Rneg)$-rule with the conclusion clause $\ol{A}$, suppose one of the clauses $\ol{B}$ in the premise is \textit{not} purely binary.
  Then either $\ol{B}$ itself is not binary, or the proof of $\ol{B}$ contains a non-binary clause $\ol{X}$.
  Since both $\ol{B}$ and $\ol{X}$ are contained in the proof of $\ol{A}$, we get that $\ol{A}$ can not be binary derived.\\
  $(\Leftarrow)$ Given a similar instance as above with the conclusion clause $\ol{A}$, suppose $\ol{A}$ is not binary-derivable.
  Then there exists a non-binary clause $\ol{Y}$ in the proof of $\ol{A}$.
  $\ol{Y}$ is either \textit{in} the premise, or it is in the proof of a clause $\ol{C}$ in the premise.  Either way, there will be a clause in the premise that is not purely binary.
\end{proof}
Even though the condition above ensures the binary derivability of the conclusion, it is however not strong enough to guarantee the purely binary property.  The following proof shows this point:
\[
\begin{prooftree}
  \Hypo{\dots}
  \Infer1{\Gamma \vdash \ol{ab}}
  \Hypo{\dots}
  \Infer1{\Gamma \vdash \ol{cd}}
  \Hypo{\dots}
  \Infer1{\Gamma \vdash \ol{ef}}
  \Infer3[$bdf$]{\Gamma \vdash \ol{ace}}
\end{prooftree}
\]
$\ol{ace}$ is obviously not purely binary in the proof above, seeing it consists of 3 atoms.  The assumption that the three clauses $\ol{ab}$, $\ol{cd}$ and $\ol{ef}$ are purely binary does not change this fact.

Since the condition in the preceding lemma coincides with the notion of binary derivability, we can simply add the binary condition to it in order to get a condition for purely binary clauses:
\begin{corollary}
  Given any instance of the $(Rneg)$-rule, the clause in the conclusion is purely binary if and only if it is binary and all the clauses in the premise are purely binary.
\end{corollary}

Using these observations, we can now define our purely binary NAND-clauses inductively:
\begin{definition}
  A NAND-clause $\ol{X}$ is purely binary iff:
  \begin{itemize}
    \item (Base Case): $\ol{X}$ is an axiom.
    \item (Inductive Case): $\ol{X}$ is binary and is the conclusion of a rule with only purely binary clauses in the premise.
  \end{itemize}
\end{definition}

TODO: Explain binary conclusion in terms of premise.

TODO: Give a corresponding graph structural definition based on the inductive definition of a purely binary clause.

TODO: Give graph examples.
\pagebreak
