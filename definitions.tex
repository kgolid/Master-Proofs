\section{Definitions and Observations}
\label{sec:Definitions and Observations}
\subsection{Graphs and Paths}
\label{sub:Graphs and Paths}
We start by writing out some basic definitions from graph theory.
\begin{definition}
  A \textbf{graph} is a tuple $G = (G_v, G_e)$ where $G_v$ is a set representing the nodes in the graph, while $G_e$ is a set representing the edges.  $G_e$ is usually represented as a subset of $G_v \times G_v$ where $(u,v) \in G_e$ iff there is an edge from node $u$ to node $v$ in the graph.
\end{definition}
\begin{definition}
  A \textbf{path} is an ordered collection of nodes $(x_0, x_1, x_2, \dots, x_n)$ from the graph such that for any two consequtive nodes $x_i,x_{i+1}$, we have that $(x_i, x_{i+1}) \in G_e$.
  In this case we say that there is a path from $x_0$ to $x_n$.
\end{definition}
\begin{definition}
  Given a node $x$ and a collection of nodes $Y$, we have that \bm{$E(x,Y)$} holds iff $x$ has outgoing edges targeting exactly the nodes in $Y$.
\end{definition}
Note that any two nodes $x,y$ such that $E(x,\{y\})$ holds will create a corresponding $xy \in OR$ in the refutation system.
\begin{definition}
  Given two nodes $x,y$ we have that \bm{$P(x,y)$} holds iff there exists a path between $x$ and $y$.
  We will use \bm{$P_o$} and \bm{$P_e$} to denote paths of odd and even lengths, respectively.
\end{definition}
  Given this definition, it is not obvious how to treat empty paths, but we will say that for any node $x$ in our graph, it is trivially true that $P_e(x,x)$ since $0$ is even.
\begin{definition}
  A path is \textbf{fully trimmed} if no nodes in the path branches off elsewhere.
  This means that for every two consecutive nodes $p,q$ in the path, we need that $E(p,\{q\})$.
\end{definition}

An \textbf{evenly trimmed} path requires this property to hold for only for every other node, starting at the first node.
We can formalize this by first giving the nodes in the path indices.
Let $x_0$ denote the first node in the path, then $x_1$ is the second node, and so on.
This lets us define the requirement for evenly trimmed path as follows: Given any two consecutive nodes $p_n,p_{n+1}$ in the path, if $n$ is even, we need that $E(p,\{q\})$.
An \textbf{oddly trimmed} path can be defined in an equivalent way.

We will denote fully, evenly and oddly trimmed paths by \bm{$P^f, P^e, P^o$}, respectively.

\subsection{Immediate Observations}
\label{sub:Immediate Observations}
