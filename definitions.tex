\section{Definitions and Observations}
\label{sec:Definitions and Observations}
\subsection{Graphs and Paths}
\label{sub:Graphs and Paths}
We start by writing out some basic definitions from graph theory.
\begin{definition}
  A \textbf{graph} is a tuple $G = (G_v, G_e)$ where $G_v$ is a set representing the nodes in the graph, while $G_e$ is a set representing the edges.
  $G_e$ is usually represented as a subset of $G_v \times G_v$ where $(u,v) \in G_e$ iff there is an edge from node $u$ to node $v$ in the graph.
\end{definition}
\begin{definition}
  A \textbf{path} is an ordered collection of nodes $(x_0, x_1, x_2, \dots, x_n)$ from the graph such that for any two consequtive nodes $x_i,x_{i+1}$, we have that $(x_i, x_{i+1}) \in G_e$.
  In this case we say that there is a path from $x_0$ to $x_n$.
\end{definition}
\subsection{Proof-specific relations}
\label{sub:Proof-specific relations}
\begin{definition}
  Given a node $x$ and a collection of nodes $Y$, we have that \bm{$E(x,Y)$} holds iff $x$ has outgoing edges targeting exactly the nodes in $Y$.
\end{definition}
\begin{definition}
  Given two nodes $x,y$ we have that \bm{$P(x,y)$} holds iff there exists a path between $x$ and $y$.
  We will use \bm{$P_o$} and \bm{$P_e$} to denote paths of odd and even lengths, respectively.
\end{definition}
\begin{definition}
  A path $P(x_0,x_n)$ is \textbf{fully trimmed} (denoted \bm{$P^f(x_0,x_n)$}) if no nodes in the path branches off elsewhere.
  This means that for every pair of consecutive nodes $x_i,x_{i+1}$ in the path, we need that $E(x_i,\{ x_{i+1} \})$.
\end{definition}
\begin{definition}
  A path $P(x_0,x_n)$ is \textbf{evenly trimmed} (denoted \bm{$P^e(x_0,x_n)$}) if every evenly indexed node is non-branching.
  Said differently, for every \textit{even} $i < n$ we need $E(x_i,\{x_{i+1}\})$ to hold.
\end{definition}
\begin{definition}
  A path $P(x_0,x_n)$ is \textbf{oddly trimmed} (denoted \bm{$P^o(x_0,x_n)$}) if for every \textit{odd} $i < n$, $E(x_i,\{x_{i+1}\})$ holds.
\end{definition}
We will usually denote paths with both length and trim in combination, i.e. $\peo(x,y)$.
This is solely to simplify the proofs and is nothing more than an abbreviation for $P^e(x,y) \wedge P_o(x,y)$, meaning an evenly trimmed odd path from $x$ to $y$.
\begin{definition}
  There is a \textbf{vel} between two nodes $x$ and $y$ (denoted \bm{$V(x,y)$}) if and only if there exists a node $c$ such that $P^o_e(x,c)$ and $P^o_o(y,c)$ and the sum of the two path lengths is odd.
\end{definition}
\subsection{Immediate observations}
\label{sub:Immediate observations}
\begin{lemma}
  $P^o(x,y) \wedge P^e(x,y) \Leftrightarrow P^f(x,y)$
\end{lemma}
A path both evenly and oddly trimmed is obviously fully trimmed.
The opposite direction should be just as obvious.
\begin{lemma}
  $P_e(x,x)$ for every $x \in G_v$
\end{lemma}
The definition of paths requires every two consequtive nodes from the ordered collection to be connected by an edge.
This is trivially realized by the empty path containing only a single node.
The path is even simply because 0 is even.
\begin{lemma}
  $E(x,\{ y \}) \Rightarrow P^f_o(x,y)$
\end{lemma}
An edge is another trivial example of a path.
The path is fully trimmed since $x$ does not branch off.
In the more general case of $E(x,Y)$, we have that $P^o_o(x,y)$ holds for every $y \in Y$.
