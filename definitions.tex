\section{Definitions and Observations}
\label{sec:Definitions and Observations}
\subsection{Graphs and Paths}
\label{sub:Graphs and Paths}
We start by writing out some basic definitions from graph theory.
\begin{definition}
  A \textbf{graph} $\bm{G} = (G, N)$ is a tuple where the first element $G$ is a set representing the nodes in the graph, while the second element $N$ is a set representing the edges (neighbours).
  $N$ is usually represented as a subset of $G \times G$ where $(u,v) \in N$ iff there is an edge from node $u$ to node $v$ in the graph.
\end{definition}
Another way to think of $N$ is as a function $N: G \rightarrow \mathcal{P}(G)$ where $N(x)$ returns the set of all out-neighbours of $x$.
\begin{definition}
  A \textbf{path} is a sequence of nodes $(x_0, x_1, x_2, \dots, x_n)$ from the graph in which all nodes (except possibly the first and last) are distinct, such that for any two consequtive nodes $x_i,x_{i+1}$, we have that $(x_i, x_{i+1}) \in N$.
  In this case we say that there is a path from $x_0$ to $x_n$.
\end{definition}
For every $x$ in the graph, we have the unique empty path $(x)$ of length 0.
This path is distinct from the loop $(x,x)$.
\subsection{Proof-specific relations}
\label{sub:Proof-specific relations}
\begin{definition}
  Given a node $x$ and a collection of nodes $Y$, we have that \bm{$E(x,Y)$} holds iff $x$ has outgoing edges targeting exactly the nodes in $Y$\todo{This definition has to be changed}.
\end{definition}
\begin{definition}
  Given two nodes $x,y$ we have that \bm{$P(x,y)$} holds iff there exists a path between $x$ and $y$.
  We will use \bm{$P_o$} and \bm{$P_e$} to denote paths of odd and even lengths, respectively.
\end{definition}
The text will sometimes use the notion of \textit{path lengths}.  This will not refer to the actual length of the path, but whether or not it is of even or odd length.
\begin{definition}
  A path $P(x_0,x_n)$ is \textbf{fully trimmed} (denoted \bm{$P^f(x_0,x_n)$}) if no nodes in the path branches off elsewhere.
  This means that for every pair of consecutive nodes $x_i,x_{i+1}$ in the path, we need that $E(x_i,\{ x_{i+1} \})$.
\end{definition}
\begin{definition}
  A path $P(x_0,x_n)$ is \textbf{evenly trimmed} (denoted \bm{$P^e(x_0,x_n)$}) if every evenly indexed node is non-branching.
  Said differently, for every \textit{even} $i < n$ we need $E(x_i,\{x_{i+1}\})$ to hold.
\end{definition}
\begin{definition}
  A path $P(x_0,x_n)$ is \textbf{oddly trimmed} (denoted \bm{$P^o(x_0,x_n)$}) if for every \textit{odd} $i < n$, $E(x_i,\{x_{i+1}\})$ holds.
\end{definition}
We will usually denote paths with both length and trim in combination, i.e. $\peo(x,y)$.
This is solely to simplify the proofs and is nothing more than an abbreviation for $P^e(x,y) \wedge P_o(x,y)$, meaning an evenly trimmed odd path from $x$ to $y$.
\begin{definition}
  There is a \textbf{vel} between two nodes $x$ and $y$ (denoted \bm{$V(x,y)$}) if and only if there exists a node $c$ such that $P^o(x,c)$ and $P^o(y,c)$ and the sum of the two path lengths is odd. In formal terms we have that
  \[
  V(x,y) \Leftrightarrow (\poo(x,c) \wedge \poe(y,c)) \vee (\poe(x,c) \wedge \poo(y,c))
  \]
\end{definition}
By introducing a bit more notation, we will be able to write this formal defintion of a vel in a nicer way.
By abstracting over the actual lengths and trims of paths we can make statements like \[P_x(a,b) \wedge P_x(b,c) \Rightarrow P_e(a,c) \quad\text{where}\quad x \in \{e,o\}\]
This lets us argue about paths of equal length (even or odd) in one statement instead of explicitly describing both cases.
The statements above tells us that the concatenation of two paths of equal length (even or odd) results in an even path\todo{Rewrite section. Use P+E examples instead of P+P}.

Similarly, we can argue about paths of different lengths by introducing $\overline{x}$ as a piece of notation such that $\overline{o} = e$ and $\overline{e} = o$.
We can now make statements like \[P_x(a,b) \wedge P_{\overline{x}}(b,c) \Rightarrow P_o(a,c)\] telling us that concatenating two paths of different lengths (even or odd) results in an odd path.
We are now able to write our definition of a vel in a nice, shorter way\todo{Does the definition below need an existencial quantifier?}:
\[
V(a,b) \Leftrightarrow \exists x(\pox(a,c) \wedge \ponx(b,c))
\]
\subsection{Immediate observations}
\label{sub:Immediate observations}
\begin{lemma}
  $P^o(x,y) \wedge P^e(x,y) \Leftrightarrow P^f(x,y)$
\end{lemma}
A path both evenly and oddly trimmed is obviously fully trimmed.
The opposite direction should be just as obvious.
\begin{lemma}
  $P_e(x,x)$ for every $x \in G_v$
\end{lemma}
This will denote the empty path mentioned earlier.  The path is even simply because 0 is even.
\begin{lemma}
  $E(x,\{ y \}) \Rightarrow P^f_o(x,y)$
\end{lemma}
An edge is another trivial example of a path.
The path is fully trimmed since $x$ does not branch off.
In the more general case of $E(x,Y)$, we have that $P^o_o(x,y)$ holds for every $y \in Y$.
\begin{lemma}
  $\poo(x,y) \Rightarrow V(x,y)$
\end{lemma}
A vel needs two paths of even and odd length, respectively.
Since the even path can be empty, we get that any odd path is a vel by itself.
\subsection{Path composition}
\label{sub:Path composition}
Given two paths such that the last node in the first path matches the first node in the second path, we have that there is a path from the first node of the first path all the way to the last node of the second path.\todo{Poorly worded.}
Using our notation, we can write:
\[
P(a,b) \wedge P(b,c) \Rightarrow P(a,c)
\]
Let's extend this statement in order to describe the change in length.
We observe that when composing two even -- or two odd -- paths, the resulting path will always be even, but when composing one path of even length with one of odd length, the resulting path will be odd.  Introducing length to the equation thus gives us two cases:
\begin{align*}
  P_x(a,b) \wedge P_x(b,c) \Rightarrow P_e(a,c)\\
  P_x(a,b) \wedge P_{\overline{x}}(b,c) \Rightarrow P_o(a,c)
\end{align*}
In order to extend these statements any further -- by introducing trims -- we need to take a look at our different paths:
\[
  \item \begin{tikzpicture}
    [
    point/.style={thick,circle,draw,inner sep=0pt,minimum size=2mm},
    trimmed/.style={thick,draw=red},
    untrimmed/.style={thick,draw=black}
    ]
    \node (0) at (0,0) [point,label=left:$\poo\quad$] ] {};
    \node (1) at (1,0) [point,trimmed] {};
    \node (2) at (2,0) [point] {};
    \node (3) at (3,0) [point,trimmed] {};
    \node (4) at (4,0) [point] {};
    \node (5) at (5,0) [point] {};
    \draw [-latex,untrimmed] (0) to (1);
    \draw [-latex,trimmed] (1) to (2);
    \draw [-latex,untrimmed] (2) to (3);
    \draw [-latex,trimmed] (3) to (4);
    \draw [-latex,untrimmed] (4) to (5);

    \node (0) at (0,1) [point,trimmed,label=left:$\peo\quad$] ] {};
    \node (1) at (1,1) [point] {};
    \node (2) at (2,1) [point,trimmed] {};
    \node (3) at (3,1) [point] {};
    \node (4) at (4,1) [point,trimmed] {};
    \node (5) at (5,1) [point] {};
    \draw [-latex,trimmed] (0) to (1);
    \draw [-latex,untrimmed] (1) to (2);
    \draw [-latex,trimmed] (2) to (3);
    \draw [-latex,untrimmed] (3) to (4);
    \draw [-latex,trimmed] (4) to (5);

    \node (0) at (0,2) [point,label=left:$\poe\quad$] ] {};
    \node (1) at (1,2) [point,trimmed] {};
    \node (2) at (2,2) [point] {};
    \node (3) at (3,2) [point,trimmed] {};
    \node (4) at (4,2) [point] {};
    \draw [-latex,untrimmed] (0) to (1);
    \draw [-latex,trimmed] (1) to (2);
    \draw [-latex,untrimmed] (2) to (3);
    \draw [-latex,trimmed] (3) to (4);

    \node (0) at (0,3) [point,trimmed,label=left:$\pee\quad$] ] {};
    \node (1) at (1,3) [point] {};
    \node (2) at (2,3) [point,trimmed] {};
    \node (3) at (3,3) [point] {};
    \node (4) at (4,3) [point] {};
    \draw [-latex,trimmed] (0) to (1);
    \draw [-latex,untrimmed] (1) to (2);
    \draw [-latex,trimmed] (2) to (3);
    \draw [-latex,untrimmed] (3) to (4);
  \end{tikzpicture}
\]
In the above figure, we have examples of the four possible combinations of trim and length of a path.  The trimmed nodes and their respective unique out-edges are depicted in red.
